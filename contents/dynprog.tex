\section{Dynamic Programming}
\subsection*{Fibonacci}
The key idea here is that at any given point, we only need the previous two values in the sequence to calculate the next one. So, toggle back and forth with the term we overwrite. (Initially, to calculate F(2), we can add F(0) and F(1), and then discard F(0). To calculate F(3), we add F(2) and F(1) and then discard F(1). And so on.) If this function looks complex, tracing through a few iterations might clarify what's going on.

\begin{lstlisting}[language=Java,basicstyle=\tiny]
int ultraFancyFib(int n)
{
   int [] f = new int[2];

   f[0] = 0;
   f[1] = 1;

   for (int i = 2; i <= n; i++)
      f[i%2] = f[0] + f[1];

   return f[n%2];
}
\end{lstlisting}

\subsection*{Combinations}
\lipsum[1][1-2]
\subsection*{Longest Common Subsequence (LCS DP)}
The trick here is to realize that at any given time, as we fill our row from left to right, there's only one value from the previous row that we might have to refer to that has already been overwritten. We can store that in a temporary variable at each iteration of the inner for-loop.

\begin{lstlisting}[language=Java,basicstyle=\tiny]
int lcss(String a, String b)
{
   if (a.length() < b.length())
   {
      String temp = a; a = b; b = temp;
   }

   // Note: This is counting on Java to initialize matrix[0] to 0.
   // In another language, we would have to do that manually in
   // order for this solution to work.

   int [] matrix = new int[b.length()+1];
   int diagonal;

   for (int i = 1; i <= a.length(); i++)
   {
      diagonal = 0;

      for (int j = 1; j <= b.length(); j++)
      {
         int weAreAboutToLoseThisValue = matrix[j];

         if (a.charAt(i-1) == b.charAt(j-1))
            matrix[j] = 1 + diagonal;
         else
            matrix[j] = Math.max(matrix[j-1], matrix[j]);

         diagonal = weAreAboutToLoseThisValue;
      }
   }

   return matrix[b.length()];
}
\end{lstlisting}

\subsection*{Number of Ways to Make Change}
\lipsum[1][1-2]
\subsection*{Fewest Number of Coins to Make Change}
\lipsum[1][1-2]
\subsection*{0-1 Knapsack Problem}

\begin{lstlisting}[language=Java,basicstyle=\tiny]
public static int knapsack(Treasure [] t, int c)
{
   int [][] matrix = new int[t.length + 1][c + 1];

   for (int i = 1; i <= t.length; i++)
      for (int j = 1; j <= c; j++)
         if (t[i-1].weight <= j)
            matrix[i][j] = Math.max(
               matrix[i-1][j],
               matrix[i-1][j-t[i-1].weight] + t[i-1].value
            );
         else
            matrix[i][j] = matrix[i-1][j];

   return matrix[t.length][c];
}
\end{lstlisting}

\subsection*{Floyd-Warshall's Algorithm and path reconstruction}

\begin{lstlisting}[language=Java,basicstyle=\tiny]
// Floyd-Warshall's all shortest paths algorithm.
private void allShortestPaths()
{
   int n = adj.length - 1;
   int [][] sp = new int[n+1][n+1];

   // Initialize base cases: sp(i,i,0) = 0, sp(i,j,0) = weight(i,j)
   for (int i = 1; i <= n; i++)
      for (int j = 1; j <= n; j++)
         sp[i][j] = (i == j) ? 0 : adj[i][j];

   // Bottom-up construction. Oh my gosh. So intense! So amazing!
   for (int k = 1; k <= n; k++)
      for (int i = 1; i <= n; i++)
         for (int j = 1; j <= n; j++)
            sp[i][j] = Math.min(sp[i][j], sp[i][k] + sp[k][j]);

   // Check for negative cycles.
   for (int i = 1; i <= n; i++)
      if (sp[i][i] < 0)
         return;
		
   // Assuming there are no negative cycles in the graph,
   // the shortest path between each (i, j) pair now lies in
   // the array at position sp[i][j]. You can do whatever you
   // want with that here:
   // ...

   // I'm printing the array just for confirmation that the method works.
   for (int i = 1; i <= n; i++)
      for (int j = 1; j <= n; j++)
         System.out.printf("%5d%s", sp[i][j], (j == n) ? "\n" : "");
}
\end{lstlisting}

\subsection*{Matrix Chain Multiplication}
\lipsum[1][1-2]
\subsection*{Edit Distance}
\lipsum[1][1-2]
\subsection*{Road Optimization Problem Idea}
\lipsum[1][1-2]
