\section{Greedy Algorithms}
\subsection*{Fractional Knapsack}
Goal is to maximize the value of a knapsack that can hold at most $W$ units worth
of goods from a list of items $I_1,I_2, \dots I_n$.

Each item has 2 attrs:
\begin{enumerate}
    \item A value/weight; let this be $v_i$ for item $I_i$.
    \item Weight available; let this be $w_i$ for item $I_i$.
\end{enumerate}
The algorithm is as follows:
\begin{enumerate}
    \item Sort the items by value/unit.
    \item Take as much as you can of the most expensive item left,
        moving down the sorted list. You may end up taking a fractional portion
        of the "last" item you take.
\end{enumerate}
\subsection*{Single Room Scheduling}
Given a single room to schedule, and a list of requests, the goal of this problem
is to maximize the total number of events scheduled.
Each request simply consists of the group, a start time and an end time during the day.

Here's the greedy solution:
\begin{enumerate}
    \item Sort the requests by finish time.
    \item Go through the requests in order of finish time,
        scheduling them in the room if the room is unoccupied at its start time.
\end{enumerate}
\subsection*{Multiple Room Scheduling}
Given a set of requests with start and end times, the goal here is to schedule
all events using the minimal number of rooms.
Once again, a greedy algorithm will suffice:
\begin{enumerate}
    \item Sort all the requests by start time.
    \item Schedule each event in any available empty room.
        If no room is available, schedule the event in a new room.
\end{enumerate}
\subsection*{Change}
The goal here is to give change with the minimal number of coins possible for a
certain number of cents using 1 cent, 5 cent, 10 cent, and 25 cent coins.

The greedy algorithm is to keep on giving as many coins of the
largest denomination until you the value that remains to be
given is less than the value of that denomination. Then you
continue to the lower denomination and repeat until you've
given out the correct change.
\subsection*{Huffman Coding}
For the following character frequencies:

\begin{tabular}{cc}
    Character & Frequency \\
    a & 12\\
    b & 2\\
    c & 7\\
    d & 13\\
    e & 14\\
    f & 85
\end{tabular}

Create a binary tree for each char that also stores the frequency w/ which it occurs.

\columnbreak[0]
The algorithm is as follows:
\begin{enumerate}
    \item Find the two bin trees in the list that store min freqs at their nodes.
    \item Connect these two nodes at a newly created common node that will store NO
        character but will store the sum of the freqs of all nodes connected below it.
\end{enumerate}

\resizebox{\linewidth}{!}{%
\begin{tikzpicture}[
    scale=0.4,
    node/.style ={
        rectangle,
        fill=white,
    },
    pics/n/.style args={#1/#2}{
        code={
            \node[node] (#2) {#1 '#2'};
        }
    }
    ]
    \node (9) {$9$}
        child{pic{n=2/b}}
        child{pic{n=7/c}};
    \pic[right=of 9]{n=12/a};
    \pic[right=of a]{n=13/d};
    \pic[right=of d]{n=14/e};
    \pic[right=of e]{n=85/f};
\end{tikzpicture}
}

Repeat until only one tree is left:

\resizebox{\linewidth}{!}{%
\begin{tikzpicture}[
    scale=0.4,
    level/.style={sibling distance=40em/#1},
    node/.style ={
        rectangle,
    },
    ]
    \node[node] (133) {133}
        child{
            node (48) {48}
                child{
                    node (21) {21}
                        child{
                            node (9) {9}
                                child{node[node] (b) {2 'b'}}
                                child{node[node] (c) {7 'c'}}
                        }
                        child{node[node] (a) {12 'a'}}
                }
                child{
                    node (27) {27}
                        child{node[node] (d) {13 'd'}}
                        child{node[node] (e) {14 'e'}}
                }
        }
        child{node[node] (f) {85 'f'}};

\end{tikzpicture}
}

One the tree is built, each leaf node corresponsds to a letter w/ a code.
To determine the code for a node, walk a std search path from the root to the leaf node.

For every step to the left, append a 0 to the code and for every step right, append a 1.

For the ex.\ tree we get the codes:

\begin{tabular}{cc}
    Character & Code \\
    a & 001\\
    b & 0000\\
    c & 0001\\
    d & 010\\
    e & 011\\
    f & 1
\end{tabular}
\subsubsection*{Calculating Bits Saved}
$\text{total bits}=\sum \text{char}_{\text{freq}}\cdot \text{char}_{\text{\# bits}}=\\%
\left(12\cdot 3\right)_{a}+\cdots+\left(85\cdot 1\right)_{f}=238$

Assuming the original file is storing each of the 6 chars with a 3-bit code.
Since there are 133 such characters, the total num of bits used before huffman is $3\cdot 133=399$.

$\therefore$ we saved $399 - 238 = 161$ bits.
